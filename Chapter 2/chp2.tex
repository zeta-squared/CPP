\documentclass[12pt, a4paper]{article}
\usepackage{algorithms}

\begin{document}
This document contains all the non-coding exercises of Chapter 2 of C++ Primer.
\subsubsection*{Exercise 2.1}

The different integer types vary in their minimum bit sizes and so their minimum value range:
\begin{itemize}
	\item [\texttt{short}]
		16 bits gives a value range of -32768 -- 32767.
	\item [\texttt{int}]
		16 bits gives a value range of -32768 -- 32767.
	\item [\texttt{long}]
		32 bits gives a value range of -2147483648 -- 2147483647.
	\item [\texttt{long long}]
		64 bits vies a value range of -9223372036854775808 -- 9223372036854775807.
\end{itemize}

\texttt{unsigned} types do not include negative values, that is, values $\geq 0$, for example if we consider \texttt{unsigned int} the value range is 0 -- 65536. Of course a \texttt{signed} type takes negative values.

Lastly, both \texttt{float} and \texttt{double} represent decimal precision values. The difference is in the accuracy of precision, a \texttt{float} is precise to 6 significant digits and \texttt{double} is precise to 10 significant digits.

\subsubsection*{Exercise 2.2}
They should all be \texttt{double} as they all involved decimal figures.

\subsubsection*{Exercise 2.3}
\end{document}
